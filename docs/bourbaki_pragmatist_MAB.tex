\documentclass[]{scrartcl}



\usepackage[english]{babel}
\usepackage{euscript}
\usepackage[utf8]{inputenc}

%spaziatura linee (opzionale)
%\linespread{1.2}


%%pacchetto per l'inclusione di figure postscript
\usepackage{graphics}
%%pacchetto per l'inclusione di figure eps
\usepackage[dvips]{graphicx}

%%comando per l'inserimento dell'intestazione
\pagestyle{headings} \makeatletter

%%pacchetto per l'inserimento di pezzi di codice
\usepackage{verbatim}


%%pacchetto per l'intestazione carina (riga sotto l'intestazione)
%\usepackage{fancyhdr}
%%\input{fancyhdr.sty}
%\pagestyle{fancy}
%\renewcommand{\chaptermark}[1]{\markboth{\thechapter.#1}{}}
%\renewcommand{\sectionmark}[1]{\markright{}}
%\lhead{} \rhead{} \chead{\emph{\leftmark}}

%%% per i diagrammi commutativi: 
\usepackage{pictexwd,dcpic}


%Ridefinizione per il quoziente!
\def\quotient#1#2{%
    \raise1ex\hbox{$#1$}\Big/\lower1ex\hbox{$#2$}%
}

%per modifica dei margini
%\addtolength{\textwidth}{1cm}
%\addtolength{\marginparsep}{2cm}

\usepackage{makeidx}
\usepackage{eurosym}
\usepackage{amsfonts}
\usepackage{latexsym}
\usepackage{makeidx}
\usepackage{amsmath}
\usepackage{amsthm}
\usepackage{amscd}
\usepackage{comment}
\usepackage{enumerate}
\usepackage{mathrsfs}
\usepackage[all]{xy}

\DeclareMathOperator*{\argmax}{arg\,max}
\DeclareMathOperator*{\argmin}{arg\,min}

\usepackage{url}
\usepackage[colorlinks=true, a4paper=true, pdfstartview=FitV,
linkcolor=blue, citecolor=blue, urlcolor=blue]{hyperref}


\newtheorem{theorem}{Theorem}[section]
\newtheorem{corollary}{Corollary}[theorem]
\newtheorem{lemma}[theorem]{Lemma}

\theoremstyle{definition}
\newtheorem{definition}{Definition}[section]

%opening
\title{Bourbaki vs Pragmatism \\ A methodological comparison through the multi-armed bandits problem}
\author{Sebastiano Ferraris\footnote{sebastiano.ferraris@gmail.com}}

\begin{document}

\maketitle

% ------------------------------------------------- %
\begin{abstract}
In these pages we introduce the well known multi-armed bandits problem and we propose a solution with two different mathematical methodologies: a Bourbakist approach and a pragmatic one. The Bourbakist way is concerned with the mathematical foundation upon which a formal solution is derived in the shape of an axiomatic structure. In contrast, the pragmatist approach aims at finding the shortest path towards a solution, reducing the mathematical formalisms to the bare minimum.
The article ends with a critical comparison, where pros and cons of each method are compared. \\

\noindent
If you came across this article when searching for an introduction to the multi-armed bandit problem, and not a methodological comparison, please do ignore section~\ref{se:bourbaki_perspective}. The code to create the figures and run a range of algorithms to solve the problem can be found at \href{https://github.com/SebastianoF/multi-armed-bandits-testbed}{https://github.com/SebastianoF/MAB}. 
\end{abstract}


% ------------------------------------------------- %
\section{Multi-armed bandits problem}
\label{se:intro}
Consider that you have to repeatedly choose between $K$ different possibilities, each having a cost and each a possible cash reward. For each possibility the cost is fixed and the reward is drawn from an \emph{unknown} probability distribution.

The problem of finding a strategy to maximise the reward in this setting is named multi armed bandits (MAB) after the situation of playing repeatedly at a row of $K$ slot machines (or single armed bandit). Given an initial amount of money of $\$1000$ and a costs of $\$1$ for each draw, the player has 1000 attempts to balance an exploration phase when estimating the unknown distributions of each arm, with an exploitation phase, when the acquired knowledge is used for a gain\footnote{See Thompson~\cite{thompson1933likelihood} for an early approach where the two arms are two medical treatments, Bellman~\cite{bellman1956problem} where the problem is formulated in a Bayesian perspective for two arms, and the more recent Sutton~\cite{sutton2018reinforcement}, chapter~1, for a reinforcement learning perspective.}.

The problem can be generalised to clinical or pre-clinical trials, control engineering, mechanical and software testing, stock market investments, behavioural modelling, dynamic pricing, and many more\footnote{See Bouneffouf~\cite{bf2019survey} for a survey with a list of applications of the main algorithms solving the MAB problem.}.

In the next section we present a solution \emph{a la Bourbaki}, and in section~\ref{se:pragmatic_perspective} in a more pragmatic way.

% ------------------------------------------------- %
\section{The Bourbakist perspective}
\label{se:bourbaki_perspective}

\begin{definition}
    Let $(\Omega, \mathcal{A})$ be a $\sigma$-algebra defined as a non-empty set $\Omega$ paired with a subset of its power set $\mathcal{A}$, containing the empty set and closed under numerable union and complement set. Let this be called \emph{action space}. Let $\mathcal{I}_{K} = \{1,2, \dots , K\} \subset \mathbb{N}$ be a set of indexes whose generic element $k$ is called \emph{arm}, by convention. Let $\mathcal{I}_{T} = \{1,2, \dots , T\} \subset \mathbb{N}$ be another set whose elements are called, again conventionally, \emph{time}. Let $\mathbb{R}_{+}$ the positive real axis including the zero. 
\end{definition}

The relationship between the above defined elements are given by a function $A$ defined as:
\begin{align*}
    A : \mathcal{I}_T \times \mathcal{A} &\longrightarrow \mathcal{I}_K \\
        (t, \omega) &\longmapsto A(t, \omega) = A_t(\omega)
\end{align*}
and by a function $R$, defined as:
\begin{align*}
\mathcal{R} : \mathcal{I}_T \times \mathcal{A} &\longrightarrow \mathbb{R}_{+} \\
(t, \omega) &\longmapsto R(t, \omega) = \mathcal{R}_t(\omega)
\end{align*}
Let the former be called \emph{action} and the latter be called \emph{reward}. Let 
\begin{align*}
R : \mathcal{I}_T \times \mathcal{I}_K &\longrightarrow \mathbb{R}_{+} \\
(t, \omega) &\longmapsto R(t, k) = R_t(k)
\end{align*}
another function, defined as the only possible function making the diagram below commutative.

\[
\begindc{\commdiag}[20]

% --- nodes:

% below
\obj(0,30)[R]{$ \mathbb{R}_{+} $}

% above
\obj(0,60)[Ik]{$ \mathcal{I}_K $}
\obj(-40,60)[A]{$ \mathcal{A} $}

% --- arrrows

% ortho
\mor{Ik}{R}{$R_{t}$}
\mor{A}{Ik}{$A_t$}

%  oblique
\mor{A}{R}{$\mathcal{R}_t$}

\enddc
\]
%
for each $t \in \mathcal{I}$. Let $R_t$ be called again \emph{reward}, and the difference between $\mathcal{R}$ and $R$ will be clear from the context.\\
We observe that $\mathcal{R}$ maps the events of the $\sigma$-algebra, while $R$ maps the corresponding indexes. This is an analogous of the definition of probability respect to the one of random variable and probability density function, for when the real axis is restricted to $[0,1]\subset\mathbb{R}$.

Now consider
\begin{align*}
    \mathcal{Q} : \mathcal{I}_T \times \mathcal{A} &\longrightarrow \mathbb{R}_{+} \\
        (t, \omega) &\longmapsto \mathcal{Q}(t, \omega) = \mathcal{Q}_t(\omega)
\end{align*}
the \emph{estimated reward of the action $\omega$ up to time $t$}, for $\omega = A_t^{-1}(k)$ for a fixed $k\in \mathcal{I}_K$, with the corresponding function $Q: \mathcal{I}_T \times \mathcal{I}_K \rightarrow \mathbb{R}$. It follows that $\mathcal{Q}$ is defined as an application of the mean value in a Lebesgue space over $(\Omega, \mathcal{A})$, that is now a Borel $\sigma$-algebra\footnote{
    For a foundational perspective, see Bourbaki~\cite{bourbaki2004integration}.
} as:

\begin{align}\label{def:mathcalQt}
\mathcal{Q}_t(\omega) = \mathbb{E} \left[ R_{\tau}(k)\mid k = A_{\tau}(\omega), \forall \tau \in \mathcal{I}_t \right]
\qquad 
\omega \in \mathcal{A}
\qquad
t \in \mathcal{I}_T
\end{align}
and therefore
\begin{align}\label{def:Qt}
Q_t(k) = \mathbb{E} \left[ \mathcal{R}_{\tau}(\omega)\mid \omega = A_{\tau}^{-1}(k), \forall \tau \in \mathcal{I}_t \right]
\qquad
t \in \mathcal{I}_T
\qquad
k \in \mathcal{I}_K
\end{align}
As $\mathcal{R}$ and $R$ did, also $\mathcal{Q}$ and $Q$ satisfies the commutativity of a diagram analogous to the one shown above. The notation can be simplified for brevity\footnote{
    The simplified notation is often the only notation appearing in engineering textbooks (e.g. Sutton~\cite{sutton2018reinforcement}), although this would not allow the reader to understand the subtle formalisation of assigning to an event $\omega$ its index $k$. More tragically the simplified notation makes most of the concepts introduced so far pedantic and irrelevant.
} to:
\begin{align}\label{def:mathcalQt_simple}
Q_t(k) = \mathbb{E} \left[ R_{t}(k) \mid A_{t} = k \right]
\end{align}
where the mean value is for all the time indexes up to $t$ and where the domain values of $A_t$ is clear from the context. 

\begin{definition}
    Let the \emph{total reward} $Q_{\infty}: \mathcal{I}_K \rightarrow \mathbb{R}$ be the function
    \begin{align}\label{def:mathcalQinf}
    Q_{\infty}(k) = \mathbb{E} \left[ \mathcal{Q}_{t}(\omega) \mid \omega = A^{-1}_{t}(k), \forall t \in \mathcal{I}_T \right]
    \qquad
    k \in \mathcal{I}_K 
    \end{align}
    or with the simplified notation as:
    \begin{align}\label{def:mathcalQinf_simple}
    Q_{\infty}(k) = \mathbb{E} \left[ R_{t}(k) \mid A_{t} = k \right]
    \end{align}
\end{definition}

So far we have been considering the reward and the total reward for a fixed choice of $k$. We can vary $k\in \mathcal{I}_K$ in function of the time index. So let $\mathbf{k}$ an element of
\begin{align*}
\mathcal{I}_K^T = \underbrace{\mathcal{I}_K\times \mathcal{I}_K \times \dots \times \mathcal{I}_K}_{T\text{-times}}
\end{align*}
or equivalently a function from $\mathcal{I}_T$ to $\mathcal{I}_K$. 
Definitions \ref{def:mathcalQt} and \ref{def:mathcalQt_simple} are so generalised to $Q_t:\mathcal{I}_K^T \rightarrow \mathbb{R}$ for $t\in\mathcal{I}_{\leq T}$, having defined $\mathcal{I}_{\leq T}$ any interval of positive integers between $1$ and $T$, and
\begin{align*}
Q_t(\mathbf{k}) = \mathbb{E} \left[ \mathcal{R}_{\tau}(\omega)
\mid
\omega = A^{-1}_{\tau}(\mathbf{k}_{\tau}), \forall \tau \in \mathcal{I}_t \right]
\qquad 
\mathbf{k} \in \mathcal{I}_K^T
\qquad
t \in \mathcal{I}_T
\end{align*}
and therefore 
\begin{align*}
Q_{\infty}(\mathbf{k}) = \mathbb{E} \left[ \mathcal{R}_{t}(\omega)
\mid
\omega = A^{-1}_{t}(\mathbf{k}_{t}),  \forall t \in \mathcal{I}_T \right]
\qquad 
\mathbf{k} \in \mathcal{I}_K^T
\end{align*}
The mean value computed with a Lebesgue measure, over the Borel space generated as the sets of images 
\footnote{
    We consider the definition under the accordance with the axiom of choice as in the ZFC axiomatic set theory, in order to avoid \emph{virages dangereux}. See also Bourbaki~\cite{bourbaki2004theory} and \cite{takeuti1982classes}.
} $\mathcal{R}_t(\omega)$ for all $\omega \in \mathcal{A}$ can be reformulated as:
\begin{align*}
Q_t(\mathbf{k}) 
= 
\frac
{\sum_{\tau=1}^{t} R_{\tau}(\mathbf{k}_{\tau}) \mathbf{1}_{A_\tau = \mathbf{k}_{\tau}}}
{\sum_{\tau=1}^{t} \mathbf{1}_{A_\tau = \mathbf{k}_{\tau}}}
\end{align*}
where $\mathbf{1}_{A_\tau = \mathbf{k}_{\tau}}$ equals to $1$ for when the event $\omega$ corresponding to $\mathbf{k}_{\tau}$ is mapped exactly to $\mathbf{k}_{\tau}$ through $A_t$, and $0$ for any other event.
Extending the time indexes up to infinity, and to justify the notation introduced above, where we used $\infty$ for a finite case, we have the following theorem:
\begin{theorem}\label{th:bourbaki}
Given a ring of infinite cardinality to which the time index $t$ belongs, and an Hilbert module\footnote{An algebraic structure generalising Hilbert vector spaces over the now introduced ring of time indexes. See for example \cite{bourbaki1987topological}.} to which the vector $\mathbf{k}$ belongs, it follows that
\begin{align*}
Q_{\infty}(\mathbf{k}) = \lim_{T \rightarrow \infty} Q_{T}(\mathbf{k})
\end{align*}
\end{theorem}
\begin{proof}
    Direct consequence of the definition of $Q_{\infty}$ generalised to the theorem hypothesis' extended structures.
\end{proof}

We now consider the value $\hat{k}$ that satisfies
\begin{align*}
\hat{k} = \argmax_{k \in \mathcal{I}_K} Q_t(k)
\qquad
\forall t \in \mathcal{I}_T
\end{align*}
for a constant value for each time index, as in the definition of $Q_t$ given in~\ref{def:Qt}. 
If we consider the possibility of varying the chosen arm $k$ across time, and so if we are allowed to compare different images of the function $A_t$ then $\hat{\mathbf{k}}$ is defined as
\begin{align}\label{eq:bourbaki_solution}
\hat{\mathbf{k}} 
= 
\argmax_{\mathbf{k} \in \mathcal{I}_K^{T}} Q_t(\mathbf{k})
\qquad
\forall t \in \mathcal{I}_T
\end{align}
Under the light of theorem~\ref{th:bourbaki}, and with the given definitions, we can now call the defined vector $\hat{\mathbf{k}}$ the \emph{solution of the generalised multi-armed bandits problem}.

% ------------------------------------------------- %
\section{The pragmatic perspective}
\label{se:pragmatic_perspective}


% ------------------------------------------------- %
\section{Discussion}
\label{se:outro}
In this article, we considered the example of the multi-armed bandit problem to prove the futility of the Bourbachist approach. 

Even if formally correct (excluding typos) and coherent (excluding Goedel\footnote{
    To this regard, in the article \emph{the ignorance of Bourbaki}~\cite{mathias1992ignorance}, Mathias noticed that the attempt of grounding the whole corpus of mathematics in an axiomatic sense have happened after, and in a conscious effort of ignoring the Goedel incompleteness theorems.
}), the theory provided in section~\ref{se:bourbaki_perspective} has the effect of turning a relatively simple problem into a complicated maze of concepts.

The pragmatic approach of section~\ref{se:pragmatic_perspective} show that there is no need of taking a functional perspective and to link it to several mechánemas developed in entirely different contexts. These are not just irrelevant in reaching a solution, they are also suffocating any possible creativity when facing a slightly different problem. If in doubt on this point the reader to continue the formalisation for the case where the unknown distributions are not fixed over time, to see how many pages and new definitions and diagrams are required. We also challenge the reader to implement the code to solve the problem having only the functional and algebraic definitions at hand rather than relying on the matrix point of view. None of the given definitions in the Bourbaki approach had provided any hints on how to solve the problem numerically, as, in conformity with the Bourbaki style, no examples had been provided. 

\subsection*{Other examples}

Despite many, and all of them more experienced than the author, have express their negative view about the Bourbakist's mathematics\footnote{
    Amongst the many: Arnold~\cite{arnol1998teaching}, De Finetti~\cite{de2008bruno}, Lockhart~\cite{lockhart2009mathematician}, the already cited Mathias and its follow up~\cite{mathias1998further}, Velupillai~\cite{velupillai2012bourbaki}
}, this approach is still widespread if not predominant across the mathematical community. 

There are in fact numerous examples of practical problems, whose pragmatic approach had been ruined by an overformalisation\footnote{
    Or assiomattisation, as Bruno de Finetti~\cite{de2008bruno} would have said playing on the Italian word \emph{matti}, meaning crazy.
} similar to the short one here proposed. They are useful examples, in particular to anyone who may believe, reading this article, that the author had been overzealous in writing section~\ref{se:bourbaki_perspective}, to give on purpose a negative light upon this the methodology.

The most notable one is the optimal transport (OT) theory. From being a pragmatic methodology of solving a class of optimisation problem, it had become a 500 pages book underpinned by a great amount of measure theory and Lebesgue spaces, perfectly irrelevant to solve any instance of an optimal transport problem. Comparing one of the original optimal transport theory presentation by Hitchcock~\cite{hitchcock1941distribution} and the formalised one by Villani~\cite{villani2003topics} is possible to see the extent of the damage. The first one is easy to read, understand, implement and possibly extend in several directions by anyone having an highschool mathematical education. The second one is a seemingly uncreated maze of unassailable interlinked concepts, requiring few years of academic studies only to grasp the first few pages, with no advantages in facing the problem. %It is almost impossible to be extended in any direction that a slightly different initial problem arising from practical need may pose, and it leaves no clarification whatsoever about why this perspective should be preferred upon the one proposed 60 years before. %Not surprisingly, a recent evolution of the OT by Patel~\cite{patelalternate} is rooted in the pragmatic approach, and does not even cite the Villani formalisation.

A second notable example of the Bourbaki effect, is in the domain of medical image registration. Here the aim is to solve the problem of finding the non-rigid deformation or metamorphosis between anatomies. The problem originates from the anatomical studies of shapes growth by D'Arcy Thompson~\cite{d1942growth} and pragmatically extended amongst others in Modersitzki~\cite{modersitzki2004numerical}. The Bourbaki over-formalized branch can be found in works like~\cite{younes2010shapes}, where we have to wait for 12 chapters before seeing what had motivated the formal mathematical theory developed until there. 
In this case too, there is no explanation of what are the advantages of the axiomatic approach respect to the pragmatic one, that have been published before.

The reader may say, for this specific case, that the pragmatic approach \cite{modersitzki2004numerical} does not use neither diffeomorphisms nor reproducing Kernel Hilbert spaces. And this is true, although it is not proved that these two mathematical devices can provide more accurate results than their pragmatic counterparts when implemented in practice. Instead it is true that they are computationally slower.

There are other branches of mathematics, such as algebraic topology, fuzzy logic and topological data analysis, whose detachment from the practical set of problems that had them originated, had turned them into axiomatic buildings, used by students to unlearn how to solve problems without overthinking and how to refrain from getting lost in useless details.

\subsection*{Why is Bourbaki still around?}

What the examples proposed in these pages intend to show, from a qualitative and empirical evaluation, is that the over-axiomatic version of a theory has several negative aspects respect to its counterpart.

These negative aspects can be: Bourbaki formalisation is not meant to implement in modern programming language. Its exhaustive cognitive load increases the difficulties of a generalisation for simple changes in the initial problem setting. Bourbaki does not provide any additional insight and intuition over the problem that originated the theory. The reader get used to work on useless details and pedantic discussions about notations, rather than solving the problem.

If all so negative, then why the Bourbaki approach is still around? The truth is, the Bourbaki approach is highly valued, prized, and appreciated\footnote{
    To this regard, the readers are welcome to copy-paste section~\ref{se:bourbaki_perspective} and to extend it, in the Bourbakist style, into an article whose title could be something like \emph{The multi-armed bandits for the working mathematician}. With this in hand the reader would experiment the level of appraisal such a work would receive.
}.

The work of the above mentioned Villani is highly cited and highly considered in the mathematical community. 
This shows that the over-axiomatization of a theory developed outside academia by mathematicians and scientist chasing practical problems is worth the Fields Medal.
Other highly cited uber-bourbachists works are for example Grotedniek~\cite{grothendieck2011some} and the four volumes of Mochizuki~\cite{mochizuki2012inter}\footnote{
    It may be interesting to notice the similarity between this work and the outcome of the randomized generator of maths paper at \href{https://thatsmathematics.com/mathgen/}{thatsmathematics.com/mathgen/}.
}. This last in particular had reach notoriety due to the claim of having solved the abc conjecture. The claim that the Bourbaki approach had provided a practical results, as in a pantomime, could not be verified\footnote{Nature paper Castelvecchi~\cite{castelvecchi2015biggest}.}. According to the mathematicians who had tried to verify it, \lq\lq the proof is too impenetrable to be understood\rq\rq.

\emph{J'accuse Bourbaki.} Nonetheless, there are several advantages of the Bourbachism that can be found, and that would justify the fact that this approach is still considered.

\emph{The outcome of the Bourbaki approach is inaccessible to neophytes}\footnote{
    An attempt of turning an over-formalised Bourbakist theory into something accessible for the layman, can be found in Villani~\cite{villani2003livingtheorem}. Interestingly enough, it seems that the only way the author could popularize his theory was by omitting the definitions and proofs while leaving only the main mathematical formulas alongside some autobiographical events. The result is even less accessible than the original work, proving that mathematicians can also be successful surrealist artists.
}, and it allows the existence of professors of mathematics who can not code or do not have any ability to solve any problem in practice. This seems to be the main reason that justifies the continuation of the Bourbachism, as neophytes usually are the people who have to decide where the public money goes and professors the ones who receives it.

\emph{The Bourbaki approach is good for the ego}. The pleasure of owning an elegant notebook filled with mathematical formulae written with a well thought after handwriting is a most sublime one. Unfortunately, after empirical evidence we can assure that this very same pleasure is a great obstacle in acquiring knowledge. The notebook owner, when facing a problem, will inevitably tend to abandon any scientific method and will bend the problem to the behold solution.

\emph{Bourbachism detaches completely mathematics from reality}, so it makes impossible to have any objective measurement of the quality or value of the work produced, besides the readers' opinion. This is a very unscientfic feature, and again a very useful one for whoever must defeat  more skilled competitors within the academic environment. 

The advocates of the superiority of pure mathematics\footnote{
    It is worth noticing that the academic dichotomy pure/applied maths is something that can not be found in mathematics before Bourbaki. I wonder if anyone could place any of the work of Archimedes, Gauss, Euler and any other great names of pre-modern mathematics in the category of pure or applied maths.
} have even arrived at accusing the mathematics that has anything to do with reality of being more prone of making mistakes\footnote{
    Gros~\cite{gros2019masters} had found how even professional mathematicians can be mislead by reality. And they had leveraged on this most surprising fact, for advocating to increase the detachment. After concluding that \lq\lq [...] we can't reason in a totally abstract manner\rq\rq, instead of suggesting to take into account the reality in the mathematical practice they suggested a move towards the opposite direction: \lq\lq We have to detach ourselves from our non-mathematical intuition\rq\rq~\cite{gros2019sciencedaily}.
}. We can be reassured by the fact that under every circumstance, the reality is adamant to persist in being what it is, and it is difficult to imagine anyone advocating for detaching medicine and engineering from reality.

\emph{The Bourbaki inspired work are more generalizable}. While this appear to be true at first glance, we claim that on the contrary a Bourbaki theory is less generalisable. Simply adding a new or different assumption to the problem settings would require to re-write almost from scratch all the axiomatic building to include the new input in a generalisable manner. The pragmatic way provides handle on the problem that are easy to be tuned or re-adapted to a slightly different one.

\emph{Bourbachism binds the concepts in formal structures avoiding paradoxes and counterexamples}. This is a very valid point in favour of the Bourbakist method, as the concerns of mathematics with counterexample have led to various discoveries, from Galois theory, to Fractals, and the search for counterexample is a valuable tool to have a better understanding when exploring the limits of mathematical knowledge\footnote{
    For example Procesi~\cite{procesi1977elementi}, Mandelbrot~\cite{mandelbrot1983fractal} and Gelbaum~\cite{gelbaum2003counterexamples}.
}. Although, the problem of the Bourbaki approach is the over-concern with counterexamples originating from theoretical considerations not related in any way to the sought solution. In section~\ref{se:bourbaki_perspective} we attained an algorithm that solves the problem, having never faced pathological counterexamples despite not relying on $\sigma$-algebras, Borel spaces, Lebesgue measures or even without explicit use of functions. What we came across were only some practical malices of the craft that are never learned by whoever is limiting themselves to the Bourbaki presentation of the problem.

About this last point in particular, there is a specific case where over-concerned researchers had been mislead by misquoted counterexamples: the medical imaging paper by Lorenzi~\cite{lorenzi2013geodesics} claims that there is no bijective correspondence between the space of the tangent vector fields and the one of their integral curves. This claim is made after citing a very peculiar counterexample found by Milnor on the tangent space of the circle.
Interestingly enough the conditions for the counterexample to happen are never met in the applications presented in the paper. The considered case is of $\mathbb{R}^3$, whose tangent space is $\mathbb{R}^3$ itself and the bijective correspondence is guaranteed by the theorem of existence and uniqueness for ODEs.


% ------------------------------------------------- %
\section{Conclusion}

The comparison proposed in this paper aims at showing how important is the mathematical approach in achieving a possible solution of a problem and how the academic over-formalised method can lead the mathematician into a maze without the centre, through a theory whose value can not be established by any objective criteria.

With the pragmatic way applied to the multi-armed bandits, an algorithm had been quickly implemented and tested without calling into play Borel or Lebesgue.

We also challenged the usefulness of the Bourbaki approach outside academia, and we suggest that it may be potentially harmful, as leading mathematics towards something useless for anyone besides who has to justify his position in academia. The Bourbaki proposes a generalisation that are not useful and counterexamples that do not arise in practice, and the mindset of the Bourbaki educated student can even obstaculate the pathway leading to a solution of a practical problem.

Despite we are still sure that students of the generations to come will be oblige to learn the Bourbaki sophistications to get a degree and to unlearn it as quick as possible to be of any use to society, this paper hopes to show that the Bourbaki method is around for academic reasons, and it is not what mathematical practice is about.

In conclusion, showing the comparison between the mathematical practice, and the one ruined by overthinking mathematicians, we want to show that mathematics is the application of the scientific method to understand the reality and is to find the algorithm that better solves a given problem. 

\bibliography{biblio} 
\bibliographystyle{alpha}


\end{document}
